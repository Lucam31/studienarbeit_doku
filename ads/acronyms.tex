%!TEX root = ../dokumentation.tex
\thispagestyle{plain}

\chapter*{Abkürzungsverzeichnis}
\markboth{Abkürzungsverzeichnis}{Abkürzungsverzeichnis}
%nur verwendete Akronyme werden letztlich im Dokument angezeigt
\begin{acronym}[MOSFET]
    %\setlength{\itemsep}{-\parsep}

    %Manuell nach Alphabet sortieren
    \acro{API}{Application Programming Interface (dt. Programmierschnittstelle)}
    \acro{GUI}{Graphical User Interface (dt. Grafische Benutzeroberfläche)}
    \acro{LIN}{Local Interconnect Network (dt. Lokales Interconnect-Netzwerk)}
    \acro{PDF}{Portable Document Format}
    \acro{POM}{Project Object Model (dt. Projektobjektmodell)}
    \acro{Reflog}{Reference Log (dt. Referenzprotokoll)}
    \acro{REST}{Representational State Transfer}
    \acro{SaaS}{Software as a Service (dt. Software als Dienst)}
    \acro{SDK}{Software Development Kit (dt. Software-Entwicklungskit)}
    \acro{SVN}{Subversion}
    \acro{VCS}{Version Control System (dt. Versionsverwaltungssystem)}
    \acroplural{VCS}[VCSs]{Version Control Systems (dt. Versionsverwaltungssysteme)}
    %Manuell nach Alphabet sortieren
    %VS Code -> Zeilen markieren STRG + Shift + P, nach Sort Lines Ascending suchen
\end{acronym}