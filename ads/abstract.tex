%!TEX root = ../dokumentation.tex

\pagestyle{empty}

\renewcommand{\abstractname}{Zusammenfassung}
\begin{abstract}
    Das Thema dieser Bachelorarbeit ist die Entwicklung eines Prototyps für eine neue Softwarearchitektur. Diese Architektur basiert auf dem LIN-Bussystem und soll die Kommunikation zwischen Slaves und Mastern überarbeiten. Ziel dieser Arbeit ist es, eine...

    Diese Bachelorarbeit behandelt die Entwicklung eines elektronischen Systems zur Temperaturbegrenzung von Strahlungsheizkörpern mithilfe von Thermoelementen. Ziel dieser Arbeit ist es, ein kostengünstigeres und flexibleres System zu entwickeln, das die bislang verwendeten mechanischen Temperaturbegrenzer ersetzt. Hintergrund ist die zunehmende Marktanforderung nach sensorbasierten Lösungen sowie die Notwendigkeit, den Oil-Ignition-Test der Norm UL~858 zu bestehen, der zukünftig auch für Strahlungsheizkörper verpflichtend sein könnte.

    Der Kern der Arbeit liegt in der Auswertung von Thermoelementen zur Temperaturbegrenzung der Strahlungsheizkörper. Die Arbeit umfasst die Entwicklung der hierfür notwendigen Hardware sowie die Programmierung der Software für den Mikrocontroller. Zusätzlich wird eine PC-Anwendung entwickelt, die den Systemstatus der entwickelten Steuerung visualisiert und eine Konfiguration der Temperaturbegrenzung ermöglicht.

    Abschließend werden Funktionstests des entwickelten Systems durchgeführt, gefolgt von einer Bewertung der Wirtschaftlichkeit des neuen Systems.
\end{abstract}


\renewcommand{\abstractname}{Abstract}
\begin{abstract}
    The scope of this bachelor thesis is the development of an electronic system for temperature limitation of radiant heating elements using thermocouples. The aim of this thesis is to develop a more cost-effective and flexible system to replace the mechanical temperature limiters currently in use. The motivation for this work is the increasing market demand for sensor-based solutions and the need to pass the Oil-Ignition Test of the UL~858 standard, which could also apply to radiant heaters in the future.

    The focus of this thesis is measuring the voltage from thermocouples to limit the temperature of the radiant heating elements. Furthermore, the thesis includes the development of the necessary hardware and the development of the microcontroller software. Additionally, a desktop application is developed to visualize the system status of the developed control system and to configure the temperature limits.

    Finally, functional tests of the developed system will be conducted, followed by an assessment of the economic efficiency of the new system.
\end{abstract}