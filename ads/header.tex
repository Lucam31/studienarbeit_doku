%!TEX root = ../dokumentation.tex

\usepackage[listings=false]{scrhack}

%Seitengroesse
\usepackage{fullpage}

%Zeilenumbruch und mehr
\usepackage[activate]{microtype}

% Zeichencodierung
\usepackage[utf8]{inputenc}
\usepackage[T1]{fontenc}
\usepackage{pifont}

% Zeilenabstand
\usepackage[onehalfspacing]{setspace}

% Index-Erstellung
\usepackage{makeidx}

% Lokalisierung (neue deutsche Rechtschreibung)
\usepackage[ngerman]{babel}

% Anführungszeichen 
\usepackage[babel,german=quotes]{csquotes}
%\usepackage[style=swiss]{csquotes}

% Spezielle Tabellenform fuer Deckblatt
\usepackage{longtable}
\setlength{\tabcolsep}{10pt} %Abstand zwischen Spalten
\renewcommand{\arraystretch}{1.5} %Zeilenabstand
\usepackage{tabularray}

% Grafiken
\usepackage{graphicx}
\usepackage{fancybox}
\usepackage{caption}
\usepackage{subcaption}
\captionsetup{figurename=Abb.}
\setlength{\fboxsep}{0pt}

% Mathematische Textsaetze
%\usepackage{amsmath}
%\usepackage{amssymb}
\usepackage{fdsymbol}

% Farben
\usepackage{soul}
\usepackage[dvipsnames]{xcolor}
\definecolor{LinkColor}{rgb}{0,0,0.2}
\definecolor{normalgreen}{RGB}{0,160,0}
\definecolor{ListingBackground}{rgb}{0.92,0.92,0.92}
\definecolor{codegreen}{rgb}{0,0.6,0}
\definecolor{codegray}{rgb}{0.5,0.5,0.5}
\definecolor{codepurple}{rgb}{0.58,0,0.82}
\definecolor{backcolour}{rgb}{0.95,0.95,0.97}
\definecolor{egruen}{rgb}{0,0.37647,0.4}

\definecolor{darkblue}{RGB}{0,29,188}
\definecolor{darkred}{RGB}{178,0,0}

\definecolor{plotred}{HTML}{AB0000}
\definecolor{plotblue}{HTML}{0000AB}
\definecolor{plotgreen}{HTML}{00AB00}

%!TEX root = ../dokumentation.tex

\newcommand{\pdftitel}{Entwicklung eines Modellfahrzeugs mit Fahrerassistenzfunktionen}
%\newcommand{\pdftiteldeckblatt}{\pdftitel} %ggf. einkommentieren, Zeile danach auskommentieren
\newcommand{\pdftiteldeckblatt}{Entwicklung eines Modellfahrzeugs mit Fahrerassistenzfunktionen}
\newcommand{\titel}{\pdftitel}
\newcommand{\autor}{Leander Gantert \& Luca Müller}
\newcommand{\arbeit}{Studienarbeit T3\_3100}
\newcommand{\arbeitsart}{\arbeit}
\newcommand{\martrikelnr}{3854248}
\newcommand{\kurs}{TINF23B3}
\newcommand{\datumAbgabe}{31. August 2026}
\newcommand{\abgabeort}{Karlsruhe}
\newcommand{\abschluss}{Bachelor of Science}
\newcommand{\studiengang}{Informationstechnik}
\newcommand{\dhbw}{Karlsruhe}
\newcommand{\gutachter}{Dipl.-Ing. (FH) Stefan Lehmann}
\newcommand{\zeitraum}{XX.XX.2025 - XX.XX.2026}

% Titel, Autor und Datum
\title{\titel}
\author{\autor}
\date{\datum}

% PDF Einstellungen
\usepackage[%
	pdftitle={\pdftitel},
	pdfauthor={\autor},
	pdfsubject={\arbeit},
	pdfcreator={pdflatex, LaTeX with KOMA-Script},
	pdfpagemode=UseOutlines, % Beim Oeffnen Inhaltsverzeichnis anzeigen
	pdfdisplaydoctitle=true, % Dokumenttitel statt Dateiname anzeigen.
	pdflang=de % Sprache des Dokuments.
]{hyperref} 

% (Farb-)einstellungen für die Links im PDF
\hypersetup{
	colorlinks=false, % De-/Aktivieren von farbigen Links im Dokument
  pdfborder={0 0 0}
	%linkcolor=LinkColor, % Farbe festlegen
	%citecolor=LinkColor,
	%filecolor=LinkColor,
	%menucolor=LinkColor,
	%urlcolor=LinkColor,
	%bookmarksnumbered=true % Überschriftsnummerierung im PDF Inhalt anzeigen.
}

% autoref Bezeichnungen anpassen
\addto\extrasngerman{\def\subsectionautorefname{Abschnitt}}
\addto\extrasngerman{\def\subsubsectionautorefname{Abschnitt}}

% Verschiedene Schriftarten
%\usepackage{goudysans}
%\usepackage{lmodern}
%\usepackage{libertine}
\usepackage{palatino} 

% Hurenkinder und Schusterjungen verhindern
% http://projekte.dante.de/DanteFAQ/Silbentrennung
\clubpenalty=10000
\widowpenalty=10000
\displaywidowpenalty=10000

\usepackage[babel,german=quotes]{csquotes}         % Deutsche Anführungszeichen + Zitate

\usepackage[
	backend = biber,                % Verweis auf biber
	language = auto,
	style = numeric,             % Nummerierung der Quellen mit Zahlen
        defernumbers=true,
	sorting = none,                 % none = Sortierung nach der Erscheinung im Dokument
    %nty
	sortcites = true,               % Sortiert die Quellen innerhalb eines cite-Befehls
	block = space,                  % Extra Leerzeichen zwischen Blocks
	hyperref = true,                % Links sind klickbar auch in der Quelle
	%backref = true,                % Referenz, auf den Text an die zitierte Stelle
	bibencoding = auto,
	giveninits = true,              % Vornamen werden abgekürzt
	doi=false,                      % DOI nicht anzeigen
	isbn=true,                     % ISBN anzeigen
    alldates=short                  % Datum immer als DD.MM.YYYY anzeigen
]{biblatex}
\addbibresource{ArbeitBib.bib}
\newcommand*{\quelle}[1]{\par\raggedleft\footnotesize Quelle:~#1}
\setcounter{biburlnumpenalty}{3000}     % Umbruchgrenze für Zahlen
\setcounter{biburlucpenalty}{6000}      % Umbruchgrenze für Großbuchstaben
\setcounter{biburllcpenalty}{9000}      % Umbruchgrenze für Kleinbuchstaben
\DeclareNameAlias{default}{family-given}  % Nachname vor dem Vornamen
\AtBeginBibliography{\renewcommand{\multinamedelim}{\addslash\space
}\renewcommand{\finalnamedelim}{\multinamedelim}}  % Schrägstrich zwischen den Autorennamen
\DefineBibliographyStrings{german}{
  urlseen = {Abruf:},                      % Ändern des Titels von "besucht am"
}
% Befehl zur Anpassung der Titeldarstellung
\DeclareFieldFormat[article, inbook, incollection, inproceedings, misc, thesis, unpublished]{title}{\normalfont\textbf{#1}}
% Befehl zur Anpassung der Titeldarstellung für Bücher
\DeclareFieldFormat[book]{title}{\normalfont\textbf{#1}}

% Quellcode
\usepackage{listings}

\lstset{
    captionpos=b, % Position der Caption (z.B. unterhalb)
    caption=\lstname, % Text der Caption (z.B. der Dateiname)
    basicstyle=\ttfamily\scriptsize,
    showstringspaces=false,
    tabsize=4,
    breaklines=true,
    frame=none,
    backgroundcolor=\color{egruen!5},
    aboveskip=20pt,
    numbers=left,
    numberstyle=\color{egruen}\scriptsize,
    stepnumber=1,
    numbersep=5pt
}
%Umlaute
\lstset{literate=%
  {Ö}{{\"O}}1
  {Ä}{{\"A}}1
  {Ü}{{\"U}}1
  {ß}{{\ss}}1
  {ü}{{\"u}}1
  {ä}{{\"a}}1
  {ö}{{\"o}}1
}

%own colors
\definecolor{owndarkpurple}{HTML}{530087}
\definecolor{owntuerkis}{HTML}{07afb8}
\definecolor{ownorange}{HTML}{cf6c02}
\definecolor{owndarkgreen}{HTML}{157302}
\definecolor{owndarktuerkis}{HTML}{029c78}

%Python
\lstnewenvironment{python}[1][]{
  \lstset{
    language=Python,
    keywordstyle=\color{blue},
    morekeywords=[1]{if, else, for, in, elif, while, break, try, except, pass, as},
    keywordstyle=[1]\color{owndarkpurple},
    morekeywords=[2]{False, True, self, polyval, argmax, median, mean, polyfit, poly1d, append, array, linspace, subplots, scatter, plot, spines, tight\_layout, tick\_params, savefig, log, annotate, legend, grid, set\_xscale, ylim, show},
    keywordstyle=[2]\color{blue},
    morekeywords=[3]{cv2, sys, os, time, np, int, str, float, ESP, ESPThread, threading, serial, matplotlib, pyplot, plt, pandas, pd, CSV, numpy, Line2D, lines, backends, backend_pdf, scipy, optimize, curve\_fit, PdfPages, pdf, log\_func\_curved},
    keywordstyle=[3]\color{owndarktuerkis},
    stringstyle=\color{ownorange},
    commentstyle=\color{owndarkgreen},
    #1
  }
}{}

%C_Code
\lstnewenvironment{c_code}[1][]{
  \lstset{
    language=C,
    keywords={unsigned, long, bool, void, int, const, CapSensor, Color, Led, float, MenuElement, SubMenu, MainMenu, Menu, VarsReactionGame, DFPlayerMini_Fast, Settings, ControlFlags, hw_timer_t, Motor, String, Temperature, HotLed, Thermocouple, Relay, SerialConnection, uint8\_t, uint16\_t,SerialConnectionManager, MeasurementManager, HeatingElement, hw\_timer\_t, false, true},
    keywordstyle=\color{blue},
    morekeywords=[2]{if, else, for, elif, while, break, switch, case, struct, new},
    keywordstyle=[2]\color{owndarkpurple},
    stringstyle=\color{ownorange},
    commentstyle=\color{owndarkgreen},
    #1
  }
}{}

% Glossar
\usepackage[
	nonumberlist, %keine Seitenzahlen anzeigen
	%acronym,      %ein Abkürzungsverzeichnis erstellen
	%section,     %im Inhaltsverzeichnis auf section-Ebene erscheinen
	toc,          %Einträge im Inhaltsverzeichnis
]{glossaries}

%Akronyme
\usepackage[printonlyused,footnote]{acronym}
% Definiere neuen Zeilenabstand für die acronym-Umgebung
\AtBeginEnvironment{acronym}{\setlength{\baselineskip}{1.15\baselineskip}}

% Fussnoten
\usepackage[perpage, hang, multiple, stable]{footmisc}

%Bildpfad
\graphicspath{{images/}}

\DeclareTOCStyleEntries[indent=0pt, numwidth=3.0em]{tocline}{figure,table}

%nur ein latex-Durchlauf für die Aktualisierung von Verzeichnissen nötig
\usepackage{bookmark}

%Gleitumgebungen (Bilder, Tabellen, usw\ldots) lassen sich mit H an genau der
% definierten Stelle platzieren
\usepackage{float}

% für die vertikale Platzierung von Text in Tabellen
\usepackage{array}

% für die Darstellung des Euro-Symbols
\usepackage[right]{eurosym}

% für textumflossene Grafiken
\usepackage{wrapfig}

\usepackage[font={color=egruen,footnotesize}, labelfont={color=egruen},
  labelsep=space]{caption}

% eine Kommentarumgebung "k" (Handhabe mit \begin{k}<Kommentartext>\end{k},
% Kommentare werden rot gedruckt). Wird \% vor excludecomment{k} entfernt,
% werden keine Kommentare mehr gedruckt.
\usepackage{comment}
\specialcomment{k}{\begingroup\color{red}}{\endgroup}
%\excludecomment{k}
\addtokomafont{disposition}{\color{egruen}}

\usepackage[automark, headsepline]{scrlayer-scrpage}
\clearpairofpagestyles
\pagestyle{scrheadings}
\renewcommand{\chapterpagestyle}{scrheadings}
\definecolor{egruen}{rgb}{0,0,0}
\ihead{
	\parbox[b]{0.63\textwidth}{\raggedright\textcolor{egruen}{\headmark}}
}
\ohead{
    
    \includegraphics[height=20pt]{images/logos/dhbw2.jpg}
}
\cfoot{\textcolor{egruen}{\thepage}}
\setlength{\headsep}{24pt} %Abstand nach Kopfzeile


%own Packages
\usepackage{pdfpages}
\usepackage{pdflscape}
\usepackage{textcomp}
\usepackage{hyperref}
\usepackage[figure]{hypcap}
\usepackage{makecell}
\usepackage[htt]{hyphenat}
\usepackage{circuitikz} %Schaltungen
\usepackage{pgfplots} %Plots/Graphs
\pgfplotsset{compat=newest}
\usepackage{ulem} %doppelt unterstreichen
\usepackage{colortbl}
\usepackage{multirow} %Zellen vertikal verbinden Tabelle
\usepackage{tablefootnote}
\usepackage{upgreek}
\usepackage{titletoc}
\usepackage{siunitx}
\sisetup{locale = DE}
\usepackage{enumitem} %itemize mit eigenem abstand
\usepackage{xurl} %URL (Umbruch)


%Anhangsverzeichnis
\DeclareNewTOC[%
  owner=\jobname,
  listname={Anhang},% Titel des Verzeichnisses
]{atoc}% Dateierweiterung (a=appendix, toc=table of contents)
\DeclareNewTOC[%
  listname={Abbildungen im Anhang},% Titel des Verzeichnisses
]{alof}% Dateierweiterung (a=appendix, lof=list of figures)
\DeclareNewTOC[%
  listname={Tabellen im Anhang},% Titel des Verzeichnisses
]{alot}% Dateierweiterung (a=appendix, lot=list of tables)
\DeclareNewTOC[%
  listname={Listings im Anhang},% Titel des Verzeichnisses
]{alol}% Dateierweiterung (a=appendix, lol=list of listings)

%Anhang
\makeatletter
\newcommand*{\useappendixtocs}{%
  \renewcommand*{\ext@toc}{atoc}%
  \scr@ifundefinedorrelax{hypersetup}{}{% damit es auch ohne hyperref funktioniert
    \hypersetup{bookmarkstype=atoc}%
  }%
  \renewcommand*{\ext@figure}{alof}%
  \renewcommand*{\ext@table}{alot}%
  \addtocontents{atoc}{\protect\renewcommand{\protect\@pnumwidth}{4em}} % Passe hier die Breite an
  \addtocontents{atoc}{\protect\renewcommand{\protect\@tocrmarg}{5em}} % Passe hier die Breite für die Kapiteltitel an (weniger = weiter nach rechts)
}
\newcommand*{\usestandardtocs}{%
  \renewcommand*{\ext@toc}{toc}%
  \scr@ifundefinedorrelax{hypersetup}{}{% damit es auch ohne hyperref funktioniert
    \hypersetup{bookmarkstype=toc}%
  }%
  \renewcommand*{\ext@figure}{lof}%
  \renewcommand*{\ext@table}{lot}%
  \addtocontents{toc}{\protect\renewcommand{\protect\@pnumwidth}{2em}} % Passe hier die Breite an
}

\xapptocmd\appendix{%
  %\addpart{\appendixname}
  \useappendixtocs
  \listofatocs %Anhangsverzeichnis
  %\listofalofs
  %\listofalots
  %\listofalols %funktioniert irgendwie nicht
}{}{}
\makeatother

\newcommand\invisiblesection[1]{%
  \refstepcounter{section}%
  \addcontentsline{atoc}{section}{\protect\numberline{\thesection}#1}%
  \sectionmark{#1}}


%Sonstige Einstellungen / eigene commands
\renewcommand*{\chapterheadstartvskip}{\vspace*{1.5\baselineskip}}% Abstand vor Überschrift einstellen
\renewcommand*{\autodot}{} % Entfernt den Punkt nach der Nummerierung

\newcommand*{\quelleBild}[1]{\par\raggedleft\footnotesize Quelle:~#1}
\newcommand{\refanhang}[1]{\hyperref[#1]{Anhang (Abschnitt \ref{#1})}} %Referenz z.B. "Anhang (Abschnitt A)"
\newcommand{\circnumb}[1]{\ding{\numexpr191+#1\relax}} %Eingekreiste Zahlen 1-10
\newcommand{\smallurl}[1]{\fontsize{7pt}{8pt}\selectfont\url{#1}} %kleiner Link (Schriftgröße, Zeilenabstand)
\newcolumntype{P}[1]{>{\raggedright\arraybackslash}p{#1}} %Text linksbündig in Tabellen mit festerbreite durch z.b. P{4cm}
\newcommand{\coloredul}[3][black]{
  \begingroup
  \definecolor{tempcolor}{HTML}{#2}
  \setulcolor{tempcolor}
  \ul{#3}
  \endgroup
}