\chapter{Grundlagen}
\label{chap:grundlagen}
In diesem Kapitel werden die nötigen Grundlagen und Technologien erläutert, die für das Verständnis dieser Arbeit erforderlich sind.

\section{Python}


\section{Hardware}
Für die Umsetzung des Projekts werden verschiedene Hardwarekomponenten verwendet, um die geplanten Funktionalitäten zu realisieren.
    \subsection{Ultraschallsensor}
    Ein Ultraschallsensor nutzt hochfrequente Schallwellen, um die Entfernung zu einem Objekt zu messen. Der Sensor besteht aus einem Sender, sowie einem Empfänger. Der Sender sendet Schallwellen aus, die von einem Objekt reflektiert werden und nach deren Rückkehr vom Empfänger aufgenommen werden. Daraufhin wird die Zeit gemessen, die die Schallwellen benötigt haben, um zum Objekt und zurück zum Sensor zu gelangen. Anhand dieser Zeit und der bekannten Schallgeschwindigkeit kann die Entfernung zum Objekt berechnet werden. %\cite{ultraschall} (this bitch ain't existing yet)
    \begin{figure}[H]
        \centering
        \includegraphics[width=0.48\textwidth]{images/02Grundlagen/ultraschallsensor.png}
        %\vspace{-0.3cm}
        \caption{Funktionsweise Ultraschallsensor}
        \vspace{-0.3cm}
        \label{fig:ultraschallsensor}
    \end{figure}

    \subsection{Raspberry Pi}
    Ein Raspberry Pi ist ein kleiner, kostengünstiger Einplatinencomputer. 

    \subsection{Motoren}


    \subsection{Raspberry Pi Camera Module}
    Was über MP, FOV, autofokus und so schreiben?

    \subsection{Stromversorgung}
    glaub nicht nötig

    \subsection{Gehäuse}


% \cite[vgl.][]{atlassian-sdk}



- tools
- Programmiersprache:
    - Bibliotheken
    - Frameworks
- Technologien
- Hardware
    - Ultraschallsensor
    - Raspberri Pi
    - Motoren
    - Kamera
    - etc.
- Protokolle
    - gRPC
    - HTTP
- etc.