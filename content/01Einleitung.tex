\chapter{Einleitung}
\label{chap:einleitung}
\section{Motivation}
In den letzten Jahren hat die Entwicklung von Fahrassistenzsystemen und autonomen Fahren erheblich zugenommen und ist mittlerweile ein zentrales Thema in der Automobilindustrie. Für viele Autofahrer sind diese Technologien bereits zu einem unverzichtbaren Bestandteil ihres Fahrerlebnisses geworden. \\
Fahrassistenzsysteme bieten eine Vielzahl an Vorteilen. Sie erhöhen die Sicherheit im Straßenverkehr, indem sie den Fahrer in kritischen Situationen unterstützen und Unfälle verhindern können. Zudem tragen sie zur Reduzierung der Belastung des Fahrers bei, da sie repetitive Aufgaben übernehmen können und so die Ermüdung verringern. \\
Mithilfe von Sensoren und Kamerasystemen können Fahrassistenzsysteme die Umgebung des Fahrzeugs überwachen und auf potenzielle Gefahren reagieren. Dies ermöglicht es dem Fahrer, sich auf andere Aspekte des Fahrens zu konzentrieren und gleichzeitig ein höheres Maß an Sicherheit zu gewährleisten. \\

\section{Problemstellung}
Da diese Systeme oft eine hohe Komplexität aufweisen und in Gefahrensituationen fehlerfrei funktionieren müssen, ist ihre Entwicklung und Implementierung eine große Herausforderung.
Bevor solche Fahrassistenzsysteme in realen Fahrzeugen implementiert werden können, ist es wichtig, diese in kontrollierten Umgebungen ausgiebig zu testen. Doch selbst in solchen Testumgebungen können Fehler zu hohen Kosten durch Schäden an Hardware und noch schlimmer zu Gefährdung Anderer führen.
Daher ist es sinnvoll, solche Systeme zunächst in einem kleineren Maßstab zu entwickeln und zu testen.

\section{Ziel dieser Arbeit}
Ziel dieser Arbeit ist die Entwicklung eines kleinen ferngesteuerten Fahrzeuges und die Implementierung einiger grundlegender Fahrassistenzsysteme. Dabei soll das Fahrzeug in der Lage sein eine Notbremsung durchzuführen, einer Linie zu folgen und Schilder zu erkennen und darauf zu reagieren. \\
Das Fahrzeug soll über eine Weboberfläche gesteuert und überwacht werden können. Hierzu sollen Sensordaten und Kamerabilder in Echtzeit übertragen werden, um dem Benutzer eine umfassende Kontrolle über das Fahrzeug zu ermöglichen. \\
Die Fernsteuerung soll noch weiter zur Sicherheit und Benutzerfreundlichkeit während der Vorführung und des Testens dieser Fahrassistenzsysteme dienen.

% \section{Funktionsumfang}