\chapter{Hardware Aufbau}
\label{chap:analyse}
In diesem Kapitel wird der Aufbau der Hardware beschrieben

\section{Raspberry Pi}
    \subsection{Aufbau des Raspberry Pi}
    Ein Raspberry Pi ist ein Einplatinencomputer, der eine Brücke zwischen Hard- und Software bildet. Er verbindet einige der Vorteile eines herkömmlichen Computers und eines Mikrocontrollers. 
    Zum einen besitzt er konfigurierbare \ac{GPIO} Pins und Anschlüsse für Versorgungsspannung und Erdung, wodurch externe Hardwarekomponenten direkt an den Raspberry Pi angeschlossen werden können. Zum anderen ist er leistungsstärker als ein Mikrocontroller. Das neuste Modell ist ausgestattet mit bis zu 16 Gigabytes an RAM, einem Arm-basierten 2,4 GHz Prozessor und einigen Schnittstellen wie beispielsweise einem \ac{LAN} Port, einer \ac{PCIe} Schnittstelle und sowohl \ac{USB}-A als auch \ac{USB}-C Ports. 

    \subsection{Raspberry Pi als Steuergerät}
    Als Steuergerät für das Fahrzeug wird ein Raspberry Pi verwendet. Der Raspberry Pi verfügt über ausreichend Leistung und Schnittstellen, um die Anforderungen des Projekts zu erfüllen. Ein Mikrocontroller wurde aufgrund der benötigten Rechenleistung und geplanten Features wie der Bildverarbeitung nicht in Betracht gezogen. \\
    Der Raspberry Pi ermöglicht die Ansteuerung der Motoren mit mithilfe der \ac{GPIO} Pins, die Verarbeitung der Sensordaten und die Kommunikation mit der Weboberfläche. Zudem bietet er die Möglichkeit, verschiedene Bibliotheken und Frameworks zu nutzen, um die Implementierung




% \cite[vgl.][Kapitel 1.3 Getting Started - What is Git?]{gitbook}


- Raspberry Pi als Steuergerät
- Ultraschallsensoren zur Hinderniserkennung
- Motoren zur Fortbewegung
- Stromversorgung
- Gehäuse
- Kamera zur visuellen Unterstützung/Schilderkennung