%%**************************************************************
%% Vorlage fuer Bachelorarbeiten (o.ä.) der DHBW
%%
%% Autor: Tobias Dreher, Yves Fischer
%% Datum: 06.07.2011
%%
%% Autor: Michael Gruben
%% Datum: 15.05.2013
%%**************************************************************
\documentclass[%
	enabledeprecatedfontcommands,
	pdftex,
	oneside,		% Einseitiger Druck.
	12pt,			% Schriftgroesse
	parskip=half,	% Halbe Zeile Abstand zwischen Absätzen.
	headsepline,	% Linie nach Kopfzeile.
	footsepline,	% Linie vor Fusszeile.
	abstracton,	    % Abstract Überschriften
	ngerman,		% Translator
	bibliography=totoc
]{scrreprt}


%!TEX root = ../dokumentation.tex

\usepackage[listings=false]{scrhack}

%Seitengroesse
\usepackage{fullpage}

%Zeilenumbruch und mehr
\usepackage[activate]{microtype}

% Zeichencodierung
\usepackage[utf8]{inputenc}
\usepackage[T1]{fontenc}
\usepackage{pifont}

% Zeilenabstand
\usepackage[onehalfspacing]{setspace}

% Index-Erstellung
\usepackage{makeidx}

% Lokalisierung (neue deutsche Rechtschreibung)
\usepackage[ngerman]{babel}

% Anführungszeichen 
\usepackage[babel,german=quotes]{csquotes}
%\usepackage[style=swiss]{csquotes}

% Spezielle Tabellenform fuer Deckblatt
\usepackage{longtable}
\setlength{\tabcolsep}{10pt} %Abstand zwischen Spalten
\renewcommand{\arraystretch}{1.5} %Zeilenabstand
\usepackage{tabularray}

% Grafiken
\usepackage{graphicx}
\usepackage{fancybox}
\usepackage{caption}
\usepackage{subcaption}
\captionsetup{figurename=Abb.}
\setlength{\fboxsep}{0pt}

% Mathematische Textsaetze
%\usepackage{amsmath}
%\usepackage{amssymb}
\usepackage{fdsymbol}

% Farben
\usepackage{soul}
\usepackage[dvipsnames]{xcolor}
\definecolor{LinkColor}{rgb}{0,0,0.2}
\definecolor{normalgreen}{RGB}{0,160,0}
\definecolor{ListingBackground}{rgb}{0.92,0.92,0.92}
\definecolor{codegreen}{rgb}{0,0.6,0}
\definecolor{codegray}{rgb}{0.5,0.5,0.5}
\definecolor{codepurple}{rgb}{0.58,0,0.82}
\definecolor{backcolour}{rgb}{0.95,0.95,0.97}
\definecolor{egruen}{rgb}{0,0.37647,0.4}

\definecolor{darkblue}{RGB}{0,29,188}
\definecolor{darkred}{RGB}{178,0,0}

\definecolor{plotred}{HTML}{AB0000}
\definecolor{plotblue}{HTML}{0000AB}
\definecolor{plotgreen}{HTML}{00AB00}

%!TEX root = ../dokumentation.tex

\newcommand{\pdftitel}{Entwicklung eines Modellfahrzeugs mit Fahrerassistenzfunktionen}
%\newcommand{\pdftiteldeckblatt}{\pdftitel} %ggf. einkommentieren, Zeile danach auskommentieren
\newcommand{\pdftiteldeckblatt}{Entwicklung eines Modellfahrzeugs mit Fahrerassistenzfunktionen}
\newcommand{\titel}{\pdftitel}
\newcommand{\autor}{Leander Gantert \& Luca Müller}
\newcommand{\arbeit}{Studienarbeit T3\_3100}
\newcommand{\arbeitsart}{\arbeit}
\newcommand{\martrikelnr}{3854248}
\newcommand{\kurs}{TINF23B3}
\newcommand{\datumAbgabe}{31. August 2026}
\newcommand{\abgabeort}{Karlsruhe}
\newcommand{\abschluss}{Bachelor of Science}
\newcommand{\studiengang}{Informationstechnik}
\newcommand{\dhbw}{Karlsruhe}
\newcommand{\gutachter}{Dipl.-Ing. (FH) Stefan Lehmann}
\newcommand{\zeitraum}{XX.XX.2025 - XX.XX.2026}

% Titel, Autor und Datum
\title{\titel}
\author{\autor}
\date{\datum}

% PDF Einstellungen
\usepackage[%
	pdftitle={\pdftitel},
	pdfauthor={\autor},
	pdfsubject={\arbeit},
	pdfcreator={pdflatex, LaTeX with KOMA-Script},
	pdfpagemode=UseOutlines, % Beim Oeffnen Inhaltsverzeichnis anzeigen
	pdfdisplaydoctitle=true, % Dokumenttitel statt Dateiname anzeigen.
	pdflang=de % Sprache des Dokuments.
]{hyperref} 

% (Farb-)einstellungen für die Links im PDF
\hypersetup{
	colorlinks=false, % De-/Aktivieren von farbigen Links im Dokument
  pdfborder={0 0 0}
	%linkcolor=LinkColor, % Farbe festlegen
	%citecolor=LinkColor,
	%filecolor=LinkColor,
	%menucolor=LinkColor,
	%urlcolor=LinkColor,
	%bookmarksnumbered=true % Überschriftsnummerierung im PDF Inhalt anzeigen.
}

% autoref Bezeichnungen anpassen
\addto\extrasngerman{\def\subsectionautorefname{Abschnitt}}
\addto\extrasngerman{\def\subsubsectionautorefname{Abschnitt}}

% Verschiedene Schriftarten
%\usepackage{goudysans}
%\usepackage{lmodern}
%\usepackage{libertine}
\usepackage{palatino} 

% Hurenkinder und Schusterjungen verhindern
% http://projekte.dante.de/DanteFAQ/Silbentrennung
\clubpenalty=10000
\widowpenalty=10000
\displaywidowpenalty=10000

\usepackage[babel,german=quotes]{csquotes}         % Deutsche Anführungszeichen + Zitate

\usepackage[
	backend = biber,                % Verweis auf biber
	language = auto,
	style = numeric,             % Nummerierung der Quellen mit Zahlen
        defernumbers=true,
	sorting = none,                 % none = Sortierung nach der Erscheinung im Dokument
    %nty
	sortcites = true,               % Sortiert die Quellen innerhalb eines cite-Befehls
	block = space,                  % Extra Leerzeichen zwischen Blocks
	hyperref = true,                % Links sind klickbar auch in der Quelle
	%backref = true,                % Referenz, auf den Text an die zitierte Stelle
	bibencoding = auto,
	giveninits = true,              % Vornamen werden abgekürzt
	doi=false,                      % DOI nicht anzeigen
	isbn=true,                     % ISBN anzeigen
    alldates=short                  % Datum immer als DD.MM.YYYY anzeigen
]{biblatex}
\addbibresource{ArbeitBib.bib}
\newcommand*{\quelle}[1]{\par\raggedleft\footnotesize Quelle:~#1}
\setcounter{biburlnumpenalty}{3000}     % Umbruchgrenze für Zahlen
\setcounter{biburlucpenalty}{6000}      % Umbruchgrenze für Großbuchstaben
\setcounter{biburllcpenalty}{9000}      % Umbruchgrenze für Kleinbuchstaben
\DeclareNameAlias{default}{family-given}  % Nachname vor dem Vornamen
\AtBeginBibliography{\renewcommand{\multinamedelim}{\addslash\space
}\renewcommand{\finalnamedelim}{\multinamedelim}}  % Schrägstrich zwischen den Autorennamen
\DefineBibliographyStrings{german}{
  urlseen = {Abruf:},                      % Ändern des Titels von "besucht am"
}
% Befehl zur Anpassung der Titeldarstellung
\DeclareFieldFormat[article, inbook, incollection, inproceedings, misc, thesis, unpublished]{title}{\normalfont\textbf{#1}}
% Befehl zur Anpassung der Titeldarstellung für Bücher
\DeclareFieldFormat[book]{title}{\normalfont\textbf{#1}}

% Quellcode
\usepackage{listings}

\lstset{
    captionpos=b, % Position der Caption (z.B. unterhalb)
    caption=\lstname, % Text der Caption (z.B. der Dateiname)
    basicstyle=\ttfamily\scriptsize,
    showstringspaces=false,
    tabsize=4,
    breaklines=true,
    frame=none,
    backgroundcolor=\color{egruen!5},
    aboveskip=20pt,
    numbers=left,
    numberstyle=\color{egruen}\scriptsize,
    stepnumber=1,
    numbersep=5pt
}
%Umlaute
\lstset{literate=%
  {Ö}{{\"O}}1
  {Ä}{{\"A}}1
  {Ü}{{\"U}}1
  {ß}{{\ss}}1
  {ü}{{\"u}}1
  {ä}{{\"a}}1
  {ö}{{\"o}}1
}

%own colors
\definecolor{owndarkpurple}{HTML}{530087}
\definecolor{owntuerkis}{HTML}{07afb8}
\definecolor{ownorange}{HTML}{cf6c02}
\definecolor{owndarkgreen}{HTML}{157302}
\definecolor{owndarktuerkis}{HTML}{029c78}

%Python
\lstnewenvironment{python}[1][]{
  \lstset{
    language=Python,
    keywordstyle=\color{blue},
    morekeywords=[1]{if, else, for, in, elif, while, break, try, except, pass, as},
    keywordstyle=[1]\color{owndarkpurple},
    morekeywords=[2]{False, True, self, polyval, argmax, median, mean, polyfit, poly1d, append, array, linspace, subplots, scatter, plot, spines, tight\_layout, tick\_params, savefig, log, annotate, legend, grid, set\_xscale, ylim, show},
    keywordstyle=[2]\color{blue},
    morekeywords=[3]{cv2, sys, os, time, np, int, str, float, ESP, ESPThread, threading, serial, matplotlib, pyplot, plt, pandas, pd, CSV, numpy, Line2D, lines, backends, backend_pdf, scipy, optimize, curve\_fit, PdfPages, pdf, log\_func\_curved},
    keywordstyle=[3]\color{owndarktuerkis},
    stringstyle=\color{ownorange},
    commentstyle=\color{owndarkgreen},
    #1
  }
}{}

%C_Code
\lstnewenvironment{c_code}[1][]{
  \lstset{
    language=C,
    keywords={unsigned, long, bool, void, int, const, CapSensor, Color, Led, float, MenuElement, SubMenu, MainMenu, Menu, VarsReactionGame, DFPlayerMini_Fast, Settings, ControlFlags, hw_timer_t, Motor, String, Temperature, HotLed, Thermocouple, Relay, SerialConnection, uint8\_t, uint16\_t,SerialConnectionManager, MeasurementManager, HeatingElement, hw\_timer\_t, false, true},
    keywordstyle=\color{blue},
    morekeywords=[2]{if, else, for, elif, while, break, switch, case, struct, new},
    keywordstyle=[2]\color{owndarkpurple},
    stringstyle=\color{ownorange},
    commentstyle=\color{owndarkgreen},
    #1
  }
}{}

% Glossar
\usepackage[
	nonumberlist, %keine Seitenzahlen anzeigen
	%acronym,      %ein Abkürzungsverzeichnis erstellen
	%section,     %im Inhaltsverzeichnis auf section-Ebene erscheinen
	toc,          %Einträge im Inhaltsverzeichnis
]{glossaries}

%Akronyme
\usepackage[printonlyused,footnote]{acronym}
% Definiere neuen Zeilenabstand für die acronym-Umgebung
\AtBeginEnvironment{acronym}{\setlength{\baselineskip}{1.15\baselineskip}}

% Fussnoten
\usepackage[perpage, hang, multiple, stable]{footmisc}

%Bildpfad
\graphicspath{{images/}}

\DeclareTOCStyleEntries[indent=0pt, numwidth=3.0em]{tocline}{figure,table}

%nur ein latex-Durchlauf für die Aktualisierung von Verzeichnissen nötig
\usepackage{bookmark}

%Gleitumgebungen (Bilder, Tabellen, usw\ldots) lassen sich mit H an genau der
% definierten Stelle platzieren
\usepackage{float}

% für die vertikale Platzierung von Text in Tabellen
\usepackage{array}

% für die Darstellung des Euro-Symbols
\usepackage[right]{eurosym}

% für textumflossene Grafiken
\usepackage{wrapfig}

\usepackage[font={color=egruen,footnotesize}, labelfont={color=egruen},
  labelsep=space]{caption}

% eine Kommentarumgebung "k" (Handhabe mit \begin{k}<Kommentartext>\end{k},
% Kommentare werden rot gedruckt). Wird \% vor excludecomment{k} entfernt,
% werden keine Kommentare mehr gedruckt.
\usepackage{comment}
\specialcomment{k}{\begingroup\color{red}}{\endgroup}
%\excludecomment{k}
\addtokomafont{disposition}{\color{egruen}}

\usepackage[automark, headsepline]{scrlayer-scrpage}
\clearpairofpagestyles
\pagestyle{scrheadings}
\renewcommand{\chapterpagestyle}{scrheadings}
\definecolor{egruen}{rgb}{0,0,0}
\ihead{
	\parbox[b]{0.63\textwidth}{\raggedright\textcolor{egruen}{\headmark}}
}
\ohead{
    
    \includegraphics[height=20pt]{images/logos/dhbw2.jpg}
}
\cfoot{\textcolor{egruen}{\thepage}}
\setlength{\headsep}{24pt} %Abstand nach Kopfzeile


%own Packages
\usepackage{pdfpages}
\usepackage{pdflscape}
\usepackage{textcomp}
\usepackage{hyperref}
\usepackage[figure]{hypcap}
\usepackage{makecell}
\usepackage[htt]{hyphenat}
\usepackage{circuitikz} %Schaltungen
\usepackage{pgfplots} %Plots/Graphs
\pgfplotsset{compat=newest}
\usepackage{ulem} %doppelt unterstreichen
\usepackage{colortbl}
\usepackage{multirow} %Zellen vertikal verbinden Tabelle
\usepackage{tablefootnote}
\usepackage{upgreek}
\usepackage{titletoc}
\usepackage{siunitx}
\sisetup{locale = DE}
\usepackage{enumitem} %itemize mit eigenem abstand
\usepackage{xurl} %URL (Umbruch)


%Anhangsverzeichnis
\DeclareNewTOC[%
  owner=\jobname,
  listname={Anhang},% Titel des Verzeichnisses
]{atoc}% Dateierweiterung (a=appendix, toc=table of contents)
\DeclareNewTOC[%
  listname={Abbildungen im Anhang},% Titel des Verzeichnisses
]{alof}% Dateierweiterung (a=appendix, lof=list of figures)
\DeclareNewTOC[%
  listname={Tabellen im Anhang},% Titel des Verzeichnisses
]{alot}% Dateierweiterung (a=appendix, lot=list of tables)
\DeclareNewTOC[%
  listname={Listings im Anhang},% Titel des Verzeichnisses
]{alol}% Dateierweiterung (a=appendix, lol=list of listings)

%Anhang
\makeatletter
\newcommand*{\useappendixtocs}{%
  \renewcommand*{\ext@toc}{atoc}%
  \scr@ifundefinedorrelax{hypersetup}{}{% damit es auch ohne hyperref funktioniert
    \hypersetup{bookmarkstype=atoc}%
  }%
  \renewcommand*{\ext@figure}{alof}%
  \renewcommand*{\ext@table}{alot}%
  \addtocontents{atoc}{\protect\renewcommand{\protect\@pnumwidth}{4em}} % Passe hier die Breite an
  \addtocontents{atoc}{\protect\renewcommand{\protect\@tocrmarg}{5em}} % Passe hier die Breite für die Kapiteltitel an (weniger = weiter nach rechts)
}
\newcommand*{\usestandardtocs}{%
  \renewcommand*{\ext@toc}{toc}%
  \scr@ifundefinedorrelax{hypersetup}{}{% damit es auch ohne hyperref funktioniert
    \hypersetup{bookmarkstype=toc}%
  }%
  \renewcommand*{\ext@figure}{lof}%
  \renewcommand*{\ext@table}{lot}%
  \addtocontents{toc}{\protect\renewcommand{\protect\@pnumwidth}{2em}} % Passe hier die Breite an
}

\xapptocmd\appendix{%
  %\addpart{\appendixname}
  \useappendixtocs
  \listofatocs %Anhangsverzeichnis
  %\listofalofs
  %\listofalots
  %\listofalols %funktioniert irgendwie nicht
}{}{}
\makeatother

\newcommand\invisiblesection[1]{%
  \refstepcounter{section}%
  \addcontentsline{atoc}{section}{\protect\numberline{\thesection}#1}%
  \sectionmark{#1}}


%Sonstige Einstellungen / eigene commands
\renewcommand*{\chapterheadstartvskip}{\vspace*{1.5\baselineskip}}% Abstand vor Überschrift einstellen
\renewcommand*{\autodot}{} % Entfernt den Punkt nach der Nummerierung

\newcommand*{\quelleBild}[1]{\par\raggedleft\footnotesize Quelle:~#1}
\newcommand{\refanhang}[1]{\hyperref[#1]{Anhang (Abschnitt \ref{#1})}} %Referenz z.B. "Anhang (Abschnitt A)"
\newcommand{\circnumb}[1]{\ding{\numexpr191+#1\relax}} %Eingekreiste Zahlen 1-10
\newcommand{\smallurl}[1]{\fontsize{7pt}{8pt}\selectfont\url{#1}} %kleiner Link (Schriftgröße, Zeilenabstand)
\newcolumntype{P}[1]{>{\raggedright\arraybackslash}p{#1}} %Text linksbündig in Tabellen mit festerbreite durch z.b. P{4cm}
\newcommand{\coloredul}[3][black]{
  \begingroup
  \definecolor{tempcolor}{HTML}{#2}
  \setulcolor{tempcolor}
  \ul{#3}
  \endgroup
}

\makeglossaries
\input{ads/glossary}
\pgfplotsset{compat=1.18} 

\begin{document}
    \hyphenation{Mini}%Nicht trennen
	% Deckblatt
	\begin{spacing}{1}
		%!TEX root = ../dokumentation.tex

\begin{titlepage}

    \begin{longtable}{l p{1.0cm} r}
		\vtop{\vskip0pt\hbox{\includegraphics[height=2.6cm]{images/logos/dhbw.png}}}
	\end{longtable}

	\enlargethispage{20mm}
	\begin{center}
	  \vspace*{12mm}	{\LARGE\normalfont\bfseries \pdftiteldeckblatt}\\
	  \vspace*{12mm}	{\large\normalfont\bfseries \arbeit}\\
	  \vspace*{12mm}	für die Prüfung zum\\
	  \vspace*{3mm} 	{\normalfont\bfseries \abschluss}\\
	  \vspace*{12mm}	des Studiengangs \studiengang\\
	  \vspace*{3mm} 	an der Dualen Hochschule Baden-Württemberg \dhbw\\
	  \vspace*{12mm}	von\\
	  \vspace*{3mm} 	{\large\normalfont\bfseries \autor}\\
	  \vspace*{12mm}	\datumAbgabe\\
	\end{center}
	\vfill
	\begin{spacing}{1.2}
	\begin{tabbing}
		mmmmmmmmmmmmmmmmmmmmmmmmmm     \= \kill
		\textbf{Bearbeitungszeitraum}  \>  \zeitraum\\
		\textbf{Matrikelnummer, Kurs}  \>  \martrikelnr, \kurs\\
		\textbf{Betreuer der DHBW}              \>  \gutachter\\
	\end{tabbing}
	\end{spacing}
\end{titlepage}

	\end{spacing}
	\newpage

	
	% Abstract
	%!TEX root = ../dokumentation.tex

\pagestyle{empty}

\renewcommand{\abstractname}{Zusammenfassung}
\begin{abstract}
    Das Thema dieser Bachelorarbeit ist die Entwicklung eines Prototyps für eine neue Softwarearchitektur. Diese Architektur basiert auf dem LIN-Bussystem und soll die Kommunikation zwischen Slaves und Mastern überarbeiten. Ziel dieser Arbeit ist es, eine...

    Diese Bachelorarbeit behandelt die Entwicklung eines elektronischen Systems zur Temperaturbegrenzung von Strahlungsheizkörpern mithilfe von Thermoelementen. Ziel dieser Arbeit ist es, ein kostengünstigeres und flexibleres System zu entwickeln, das die bislang verwendeten mechanischen Temperaturbegrenzer ersetzt. Hintergrund ist die zunehmende Marktanforderung nach sensorbasierten Lösungen sowie die Notwendigkeit, den Oil-Ignition-Test der Norm UL~858 zu bestehen, der zukünftig auch für Strahlungsheizkörper verpflichtend sein könnte.

    Der Kern der Arbeit liegt in der Auswertung von Thermoelementen zur Temperaturbegrenzung der Strahlungsheizkörper. Die Arbeit umfasst die Entwicklung der hierfür notwendigen Hardware sowie die Programmierung der Software für den Mikrocontroller. Zusätzlich wird eine PC-Anwendung entwickelt, die den Systemstatus der entwickelten Steuerung visualisiert und eine Konfiguration der Temperaturbegrenzung ermöglicht.

    Abschließend werden Funktionstests des entwickelten Systems durchgeführt, gefolgt von einer Bewertung der Wirtschaftlichkeit des neuen Systems.
\end{abstract}


\renewcommand{\abstractname}{Abstract}
\begin{abstract}
    The scope of this bachelor thesis is the development of an electronic system for temperature limitation of radiant heating elements using thermocouples. The aim of this thesis is to develop a more cost-effective and flexible system to replace the mechanical temperature limiters currently in use. The motivation for this work is the increasing market demand for sensor-based solutions and the need to pass the Oil-Ignition Test of the UL~858 standard, which could also apply to radiant heaters in the future.

    The focus of this thesis is measuring the voltage from thermocouples to limit the temperature of the radiant heating elements. Furthermore, the thesis includes the development of the necessary hardware and the development of the microcontroller software. Additionally, a desktop application is developed to visualize the system status of the developed control system and to configure the temperature limits.

    Finally, functional tests of the developed system will be conducted, followed by an assessment of the economic efficiency of the new system.
\end{abstract}
	\newpage

	\renewcommand{\thepage}{\Roman{page}}

	
	% Erklärung
	%!TEX root = ../dokumentation.tex

\thispagestyle{empty}

\section*{Erklärung}
% http://www.se.dhbw-mannheim.de/fileadmin/ms/wi/dl_swm/dhbw-ma-wi-organisation-bewertung-bachelorarbeit-v2-00.pdf
\vspace*{2em}

Ich versichere hiermit, dass ich meine {\arbeitsart} mit dem Thema: {\itshape \titel } selbstständig verfasst und keine anderen als die angegebenen Quellen und Hilfsmittel benutzt habe.

\vspace{3em}

\abgabeort, \datumAbgabe

\begin{figure}[H]
    \includegraphics[width=0.2\linewidth]{images/logos/Unterschrift.jpeg}
\end{figure}
\vspace{-0.8cm}
\autor

	\newpage
	
	\pagenumbering{Roman}
	\setcounter{page}{1}

	%\pagestyle{plain}
	\pagestyle{scrheadings}


	% Inhaltsverzeichnis
	\begin{spacing}{1.2}
		\setcounter{tocdepth}{2} %Wie viele Unterkapitel werden angezeigt?
		\tableofcontents
	\end{spacing}
	\newpage
    
    
    % Verzeichnisse
	\clearpage
	% \pagenumbering{Roman}
	% \setcounter{page}{8}
	
    % Abbildungsverzeichnis
	\cleardoublepage
	
	\begin{minipage}[b]{1\linewidth}
		\phantomsection \label{listoffig}
		\addcontentsline{toc}{chapter}{Abbildungsverzeichnis}
		\listoffigures
	\end{minipage}

	\begin{minipage}[b]{1\linewidth}
		\phantomsection \label{listoftab}
		\addcontentsline{toc}{chapter}{Tabellenverzeichnis}
		\listoftables
	\end{minipage}

	\begin{minipage}[b]{1\linewidth}
		\phantomsection \label{listoflist}
		\addcontentsline{toc}{chapter}{Listings}
		% Fix listings alignment right before generating the list
		\makeatletter
		\renewcommand*{\l@lstlisting}[2]{%
			\@dottedtocline{1}{0em}{2.3em}{#1}{#2}%
		}
		\makeatother
		\lstlistoflistings
	\end{minipage}

    %Tabellenverzeichnis
	% \cleardoublepage % Abb. Verz. und Tab. Verz. auf eine Seite wenn so kurz

    % Quellcodeverzeichnis
	% \cleardoublepage
	% \phantomsection \label{listoflist}
	% \addcontentsline{toc}{chapter}{Listings}
	% \lstlistoflistings

    % Abkürzungsverzeichnis
	\cleardoublepage
	\phantomsection \label{listofacs}
	\addcontentsline{toc}{chapter}{Abkürzungsverzeichnis}
	%!TEX root = ../dokumentation.tex
\thispagestyle{plain}

\chapter*{Abkürzungsverzeichnis}
\markboth{Abkürzungsverzeichnis}{Abkürzungsverzeichnis}
%nur verwendete Akronyme werden letztlich im Dokument angezeigt
\begin{acronym}[MOSFET]
    %\setlength{\itemsep}{-\parsep}

    %Manuell nach Alphabet sortieren
    \acro{API}{Application Programming Interface (dt. Programmierschnittstelle)}
    \acro{GUI}{Graphical User Interface (dt. Grafische Benutzeroberfläche)}
    \acro{LIN}{Local Interconnect Network (dt. Lokales Interconnect-Netzwerk)}
    \acro{PDF}{Portable Document Format}
    \acro{POM}{Project Object Model (dt. Projektobjektmodell)}
    \acro{Reflog}{Reference Log (dt. Referenzprotokoll)}
    \acro{REST}{Representational State Transfer}
    \acro{SaaS}{Software as a Service (dt. Software als Dienst)}
    \acro{SDK}{Software Development Kit (dt. Software-Entwicklungskit)}
    \acro{SVN}{Subversion}
    \acro{VCS}{Version Control System (dt. Versionsverwaltungssystem)}
    \acroplural{VCS}[VCSs]{Version Control Systems (dt. Versionsverwaltungssysteme)}
    %Manuell nach Alphabet sortieren
    %VS Code -> Zeilen markieren STRG + Shift + P, nach Sort Lines Ascending suchen
\end{acronym}
	
    % Symbolverzeichnis
	% \cleardoublepage
	% %!TEX root = ../dokumentation.tex
\thispagestyle{plain}
%Nomenklatur
\chapter*{Symbolverzeichnis}
\markboth{Symbolverzeichnis}{Symbolverzeichnis}
\addcontentsline{toc}{chapter}{Symbolverzeichnis}
\DeclareSIUnit{\millisecond}{ms}
\DeclareSIUnit{\litre}{l}

\begin{table}[!h]
    \renewcommand{\arraystretch}{1.5}
    \begin{tabular}{p{2cm}p{2cm}p{7cm}}
        \textbf{Symbol} & \textbf{Einheit} & \textbf{Bezeichnung}\\
        %Alphabetisch sortieren?
        %$a$             & m s$^{-2}$        & Beschleunigung\\
%        $C$             & \qty{}{\farad}    & Kapazität\\
        %$F$             & N                 & Kraft\\
%        $f$             & \qty{}{\hertz}    & Frequenz\\
%        $I$             & \qty{}{\ampere}   & Elektrischer Strom\\
        %$L$             & $H$               & Induktivität\\
%        $P$             & \qty{}{\watt}     & Leistung\\
        %$Q$             & $C$               & Ladung\\
        %$M$             & $N m$             & Drehmoment\\
        %$m$             & $kg$              & Masse\\
        %$n$             &                   & Anzahl Umdrehungen\\
%        $R$             & \qty{}{\ohm}      & Elektrischer Widerstand\\
%        $S$             & \qty{}{\volt\per\kelvin}      & Seebeck-Koeffizient\\
        %$s$             & $m$               & Strecke/Länge/Weg\\
        $t$             & \qty{}{\millisecond}, \qty{}{\second}   & Zeit\\
%        $U$             & \qty{}{\volt}     & Elektrische Spannung\\
        %$$V_T$$         & -                 & Tastgrad\\
%        $V$             &    & Verstärkung\\
        %$v$             & $m s$^{-1}$       & Geschwindigkeit\\
%        $W$             & \qty{}{\watt\second} & Elektrische Arbeit\\
%        $X$             & \qty{}{\ohm}      & Blindwiderstand\\
        %$\eta$          & -                 & Wirkungsgrad\\
        %$\upvarphi$}    & Grad ($^\circ$)   & Phasenwinkel\\
        $Q$				& \qty{}{\milli\litre\per\min} & Durchflussrate\\
        $\vartheta $    & \qty{}{\degreeCelsius} & Temperatur\\
        
        %$$             &$$                 & \\

        %Noch Alphabetisch sortieren!!!

    \end{tabular}
    \label{tab:my_label}
\end{table}

	\newpage

	\newcounter{romanpages}
	\setcounter{romanpages}{\value{page}}

    % Inhalt
	\clearpage
	\pagenumbering{arabic}
	\setcounter{page}{1}

        \chapter{Einleitung}
\label{chap:einleitung}
\section{Motivation}
%\begin{wrapfigure}{r}{0.5\textwidth}
%    \centering
%    \vspace{-2.0cm}
%    \includegraphics[width=0.48\textwidth]{images/Einleitung/Dickschicht_Produktbild.jpg}
%    %\vspace{-0.3cm}
%    \caption{Dickschichttechnologie \cite{dickschicht}}
%    \vspace{-0.3cm}
%    \label{fig:Dickschicht}
%\end{wrapfigure}
\begin{itemize}
    \item Welche persönlichen oder fachlichen Interessen haben dieses Thema inspiriert?
    \item Welche Relevanz hat das Thema in der aktuellen Forschung oder Industrie?
    \item Wie trägt diese Arbeit zur Weiterentwicklung oder Verbesserung eines bestimmten Bereichs bei?
    \item Warum ist das Thema für mich und andere von Bedeutung?
\end{itemize}

\section{Problemstellung}
\begin{itemize}
    \item Welche praktischen Anwendungen oder Vorteile könnten aus den Ergebnissen dieser Arbeit resultieren?
    \item Welche Herausforderungen oder Probleme sollen durch diese Arbeit adressiert werden?
    \item Welche Probleme treten bei der aktuellen Lösung/Situation auf? Was wird durch die Umsetzung dieses Projekts verbessert?
\end{itemize}

\section{Ziel dieser Arbeit}
\begin{itemize}
    \item Wie werden die Probleme, die in der Problemstellung identifiziert wurden, durch die Ergebnisse dieser Arbeit gelöst?
    \item Was genau wird erreicht?
\end{itemize}

\section{Funktionsumfang}
        \chapter{Grundlagen}
\label{chap:grundlagen}
In diesem Kapitel werden die nötigen Grundlagen und Technologien erläutert, die für das Verständnis dieser Arbeit erforderlich sind.

\section{Python}


\section{Hardware}
Für die Umsetzung des Projekts werden verschiedene Hardwarekomponenten verwendet, um die geplanten Funktionalitäten zu realisieren.
    \subsection{Ultraschallsensor}
    Ein Ultraschallsensor nutzt hochfrequente Schallwellen, um die Entfernung zu einem Objekt zu messen. Der Sensor besteht aus einem Sender, sowie einem Empfänger. Der Sender sendet Schallwellen aus, die von einem Objekt reflektiert werden und nach deren Rückkehr vom Empfänger aufgenommen werden. Daraufhin wird die Zeit gemessen, die die Schallwellen benötigt haben, um zum Objekt und zurück zum Sensor zu gelangen. Anhand dieser Zeit und der bekannten Schallgeschwindigkeit kann die Entfernung zum Objekt berechnet werden. %\cite{ultraschall} (this bitch ain't existing yet)
    \begin{figure}[H]
        \centering
        \includegraphics[width=0.48\textwidth]{images/02Grundlagen/ultraschallsensor.png}
        %\vspace{-0.3cm}
        \caption{Funktionsweise Ultraschallsensor}
        \vspace{-0.3cm}
        \label{fig:ultraschallsensor}
    \end{figure}

    \subsection{Raspberry Pi}
    Ein Raspberry Pi ist ein kleiner, kostengünstiger Einplatinencomputer. 

    \subsection{Motoren}


    \subsection{Raspberry Pi Camera Module}
    Was über MP, FOV, autofokus und so schreiben?

    \subsection{Stromversorgung}
    glaub nicht nötig

    \subsection{Gehäuse}


% \cite[vgl.][]{atlassian-sdk}



- tools
- Programmiersprache:
    - Bibliotheken
    - Frameworks
- Technologien
- Hardware
    - Ultraschallsensor
    - Raspberri Pi
    - Motoren
    - Kamera
    - etc.
- Protokolle
    - gRPC
    - HTTP
- etc.
        \chapter{Analyse und Anforderungen}
\label{chap:analyse}
In diesem Kapitel wird eine Analyse des bestehenden Systems durchgeführt und die Anforderungen an das neue System beschrieben. 

\section{Analyse des bestehenden Systems}

\section{Anforderungen an das neue System}

\section{Anforderungen an die GUI-Anwendung}


% \cite[vgl.][Kapitel 1.3 Getting Started - What is Git?]{gitbook}



        \chapter{Konzept der neuen Architektur}
\label{chap:architektur}

\section{Übersicht und Aufbau der Architektur}

\section{}


		\chapter{Implementierung und Integration}
\label{chap:implementierung}

		\input{content/06Testing.tex}
		\chapter{Fazit und Ausblick}
\label{fazit}
% \section*{Leitfragen für das Fazit}
% \begin{itemize}
%     \item Welche zentralen Ergebnisse wurden in der Arbeit erzielt?
%     \item Inwiefern wurden die Forschungsfragen beantwortet?
%     \item Welche Herausforderungen traten während der Arbeit auf und wie wurden sie gelöst?
%     \item Welche Bedeutung haben die Ergebnisse für das behandelte Thema?
%     \item Welche Grenzen und Einschränkungen hat die Arbeit?
% \end{itemize}

% \section*{Leitfragen für den Ausblick}
% \begin{itemize}
%     \item Welche offenen Fragen bleiben nach der Arbeit bestehen?
%     \item Welche weiterführenden Forschungsansätze ergeben sich aus den Ergebnissen?
%     \item Wie könnten die Ergebnisse in der Praxis angewendet werden?
%     \item Welche zukünftigen Entwicklungen könnten das Thema beeinflussen?
%     \item Welche Verbesserungen oder Erweiterungen wären für die Arbeit denkbar?
% \end{itemize}

- Ergebnisse als eigene Datei?

    % ---- Literaturverzeichnis
        \cleardoublepage
        \pagenumbering{Roman}           % Römische Seitenzahlen
        \setcounter{page}{\value{romanpages}}
    
    %Quellen
        %Literatur
			\sloppy
            \printbibliography[title={Literaturverzeichnis}, notkeyword={datenblatt}, notkeyword={bild}, notkeyword={bib}]
            \label{chap:literaturverzeichnis}
        %Datenblätter
            \printbibliography[title={Datenblätter}, keyword={datenblatt}, heading=subbibliography]
        %Bibliotheken
            \printbibliography[title={Softwarebibliotheken}, keyword={bib}, heading=subbibliography]
            \label{bib:bib}
        %Bilder
            \printbibliography[title={Abbildungen}, keyword={bild}, heading=subbibliography]
            %\footnote{Abbildungen ohne Quellenangabe wurden selbst erstellt.}
        

    
    %Anhang
%    \cleardoublepage
%	\phantomsection \label{anhang}
%	\addcontentsline{toc}{chapter}{Anhang}
% 
%    \appendix
%        



%\chapter{Diagramme des SHK-Abkühlvorgangs}
%\section{SHK-Varianten mit einem Thermoelement in einem Keramikgehäuse}
%\label{anhang:abkuehlnormal}
%\vspace{-0.1cm}
%\input{images/Funktionstests/GOT_normal/plot_shk2_got_210_abschalt.tex}
%\vspace{-0.5cm}
%\input{images/Funktionstests/GOT_normal/plot_shk4_got_140_abschalt.tex}
%
%\section{SHK-Varianten mit einem im Isolierring integrierten Thermoelement}
%\label{anhang:abkuehloilign}
%\input{images/Funktionstests/GOT_oil_ign_sensor/plot_shk1_got_oil_ignition_abschalt.tex}
%\vspace{-0.5cm}
%\input{images/Funktionstests/GOT_oil_ign_sensor/plot_shk3_got_oil_ignition_abschalt.tex}

		
\end{document}
